% !TeX encoding = UTF-8
\documentclass{article}
\usepackage[utf8]{inputenc}
\usepackage{amsmath}
\usepackage{amsfonts}
\usepackage{amssymb}
\usepackage{xcolor}
\usepackage{graphicx}
\usepackage{subfigure}
\usepackage{float}
\usepackage{algorithm}
\usepackage{algorithmic}
\usepackage{fontawesome}
\usepackage{longtable} % For tables that can span multiple pages
\usepackage{geometry}
\usepackage{booktabs}
% \renewcommand\thesubsection{\alph{subsection}.} % Set subsection numbering to use letters

\usepackage{geometry}
\geometry{margin=1in} % Set margin

\usepackage{hyperref}
\usepackage{pythonhighlight}
\usepackage{listings}
\lstset{
    basicstyle=\ttfamily\small,
    breaklines=true,
    frame=single,
    keywordstyle=\color{blue},
    commentstyle=\color{green!50!black},
    stringstyle=\color{red},
    numbers=left,
    numberstyle=\tiny,
    stepnumber=1,
    tabsize=2,
    showspaces=false,
    showstringspaces=false
}


\usepackage{hyperref}
\usepackage[capitalize]{cleveref}
\hypersetup{
    colorlinks=true,
    linkcolor=blue,
    filecolor=magenta,      
    urlcolor=cyan,
    pdftitle={Overleaf Example},
    pdfpagemode=FullScreen,
}
% for Chinese
\usepackage{fontspec}               		% 加這個就可以設定字體
\usepackage[BoldFont, SlantFont]{xeCJK} 

\begin{document}
\title{Spring 2025 Cloud Native Application Development HW2 \\ Report}
\author{Leonard Tsai B10705010}
\date{\today}

\maketitle
\section{Project Overview}
The Math CLI Application is a command-line interface tool that provides basic mathematical operations. The application is built with Python and follows modern software development practices including unit testing, continuous integration, and code coverage analysis.

\section{Technical Framework}
\subsection{Development Environment}
\begin{itemize}
    \item Python 3.8+
    \item pytest for unit testing
    \item GitHub Actions for CI/CD
    \item Codecov for code coverage reporting
\end{itemize}

\subsection{Project Structure}
\begin{verbatim}
math_cli_app/
├── math_cli_app/
│   ├── __init__.py
│   ├── addition.py
│   ├── factorial.py
│   ├── odd_even.py
│   └── main.py
├── unit_test/
│   ├── __init__.py
│   ├── test_addition.py
│   ├── test_factorial.py
│   └── test_odd_even.py
├── setup.py
└── requirements.txt
\end{verbatim}

\section{Core Functions}
\subsection{Addition Function}
\begin{lstlisting}[language=Python]
def add_numbers(a: float, b: float) -> float:
    """Add two numbers together."""
    return a + b
\end{lstlisting}
Features:
\begin{itemize}
    \item Supports floating-point numbers
    \item Handles positive and negative numbers
    \item Returns sum as float
\end{itemize}

\subsection{Factorial Function}
\begin{lstlisting}[language=Python]
def calculate_factorial(n: int) -> int:
    """Calculate factorial of a non-negative integer."""
    if n < 0:
        raise ValueError("Factorial not defined for negative numbers")
    if n == 0 or n == 1:
        return 1
    return n * calculate_factorial(n - 1)
\end{lstlisting}
Features:
\begin{itemize}
    \item Handles non-negative integers
    \item Raises ValueError for negative inputs
    \item Optimized for n = 0 and n = 1
\end{itemize}

\subsection{Odd/Even Function}
\begin{lstlisting}[language=Python]
def check_odd_even(n: int) -> str:
    """Check if a number is odd or even."""
    return "Even" if n % 2 == 0 else "Odd"
\end{lstlisting}
Features:
\begin{itemize}
    \item Works with integers
    \item Returns "Even" or "Odd" as string
    \item Handles negative numbers
\end{itemize}

\section{Testing Framework}
\subsection{Unit Tests}
The project includes comprehensive unit tests for all functions:
\begin{itemize}
    \item Addition tests: positive numbers, negative numbers, zero, floats
    \item Factorial tests: positive integers, zero, error cases
    \item Odd/Even tests: positive/negative integers, zero
\end{itemize}

\subsection{Continuous Integration}
GitHub Actions workflow includes:
\begin{itemize}
    \item Automated testing on Python 3.8, 3.9, and 3.10
    \item Code coverage reporting
    \item Integration with Codecov
\end{itemize}

\section{Issues and Feature Requests}
\subsection{Current Issues}
\begin{enumerate}
    \item Initial setup of unit tests
    \item Integration of code coverage reporting
    \item Implementation of CI/CD pipeline
\end{enumerate}

\subsection{Implemented Solutions}
\begin{enumerate}
    \item Created comprehensive test suite
    \item Configured GitHub Actions workflow
    \item Set up Codecov integration
\end{enumerate}

\section{Future Enhancements}
\begin{itemize}
    \item Additional mathematical functions
    \item Enhanced error handling
    \item Performance optimizations
    \item User interface improvements
\end{itemize}

\section{Security Considerations}
\begin{itemize}
    \item Input validation for all functions
    \item Error handling for edge cases
    \item Secure CI/CD pipeline configuration
\end{itemize}


\end{document}